На 28 минуте олимпиады команда Гимназии №1543 ``Суслики'' отправила на проверку решение по задаче A.``Слоники'' на языке программирования GNU C++ 4.7 (Решение №1219), которое получило вердикт ``Неправильный ответ'' (Протокол проверки №1219)
\begin{lstlisting}[language=c++, label=r1219, caption=Решение №1219]
#include <iostream>
#include <math.h>
using namespace std;
bool xy (int a)
{
    if (a == 1) return false;

    for(int i = 2; i <= sqrt(a); ++i)
    {
        if(a % i == 0)
        {
            return false;
        }
    }
    return true;
}
int main()
{
    int l,k,sum;
    cin>>l>>k;
    if(k>l)
    {
        cout<<-1;
        return 0;
    }
    for(int i=1;i<30000-l-1;i+=3)
    {
        sum=0;
        for(int j=i;j<i+l;++j)
        {
            if(xy(j))
            {
                ++sum;
            }
        }
        if(sum==k)
        {
            cout<<i<<" "<<i+l-1;
            return 0;
        }
    }
    cout<<-1;
    return 0;
}

\end{lstlisting}


Результат проверки решения №1219:

 \begin{longtable}{|p{1cm}|p{2.5cm}|p{1.5cm}|p{2.5cm}|p{1.5cm}|p{5cm}|}\hline
№ теста & Результат & ЦПУ, сек. & Общее время, сек. & Память & Комментарий по результату работы \\ \hline 
1 & OK & 1 & 2 & 286720 & \begin{spverbatim}ok ok

\end{spverbatim}  \\ \hline
2 & OK & 66 & 70 & 3502080 & \begin{spverbatim}ok ok

\end{spverbatim}  \\ \hline
3 & OK & 2 & 4 & 286720 & \begin{spverbatim}ok ok

\end{spverbatim}  \\ \hline
4 & OK & 1 & 2 & 286720 & \begin{spverbatim}ok ok

\end{spverbatim}  \\ \hline
5 & OK & 1 & 2 & 286720 & \begin{spverbatim}ok ok

\end{spverbatim}  \\ \hline
6 & WA & 1 & 1 & 286720 & \begin{spverbatim}wrong answer Integer -1 violates the range [1, 1]

\end{spverbatim}  \\ \hline
\end{longtable}