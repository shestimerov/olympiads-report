На 28 минуте олимпиады команда Гимназии №1543 ``Суслики'' отправила на проверку решение по задаче A.``Слоники'' на языке программирования GNU C++ 4.7 (Решение №1217), которое получило вердикт ``Ошибка во время выполнения программы'' (Протокол проверки №1217)
\begin{lstlisting}[language=c++, label=r1217, caption=Решение №1217]
var
D, Ap, Bp, Af, Bf :integer;
Afirst , Bfirst :integer;  //1 флаг 2 плакат
different: boolean;
m, n, l, min1, min2, answer, max1, max2 :integer;

begin
  readln(D);
  readln(Ap);
  readln(Bp);
  readln(Af);
  readln(Bf);
  
  if (D = 1) then
  begin
    if (((Af < Ap) and (Bf > Bp)) or ((Af > Ap) and (Bf < Bp))) then
    begin
      different:=true;
    end
    else
    begin
      different:=false;
    end
  end;
  if (D = -1) then
  begin
    if (((Af < Ap) and (Bf < Bp)) or ((Af > Ap) and (Bf > Bp))) then
    begin
      different:=true;
    end
    else
    begin
      different:=false;
    end
  end;
  
  if (different = true) then
  begin
    if(((Af < Ap) and (Bp < Bf)) or ((Af > Ap) and (Bp > Bf)) or ((Af < Ap) and (Bp > Bf)) or ((Af < Ap) and (Bp > Bf))) then
    begin
      m:=Af-Ap;
      if(m < 0) then
      begin
        m:=-m;
      end;
      n:=Bf-Bp;
      if(n < 0) then
      begin
        n:=-n;
      end;
      answer:=n+m
    end;
  end;
  
  
  if (different = false) then
  begin
    m:=Af-Ap;
    if(m < 0) then
    begin
      m:=-m;
    end;
    n:=Bf-Bp;
    if(n < 0) then
    begin
      n:=-n;
    end;
  
    if (Af > Ap) then
    begin
     max1:=Af
    end
    else
    begin
     max1:=Ap
    end;
    
    if (Bf > Bp) then
    begin
     max2:=Bf
    end
    else
    begin
      max2:=Bp
    end;
    
    if (max1 > max2) then
    begin
      answer:= max1
    end
    else
    begin
      answer:= max2;
    end;
    
    if (((Ap = Af) and (Bp <> Bf)) or ((Ap <> Af) and (Bp = Bf))) then
    begin
      answer:= -1;
    end;
    
    if (m<>n) then
    begin
      answer:= -1;
    end;
  end;
  
  
  
  
  
  
  writeln(answer);
end.
\end{lstlisting}


Результат проверки решения №1217:

 \begin{longtable}{|p{1cm}|p{2.5cm}|p{1.5cm}|p{2.5cm}|p{1.5cm}|p{5cm}|}\hline
№ теста & Результат & ЦПУ, сек. & Общее время, сек. & Память & Комментарий по результату работы \\ \hline 
1 & RT & 53 & 56 & 7897088 & \begin{spverbatim}  File "./001217", line 5
    D, Ap, Bp, Af, Bf :integer;
                      ^
SyntaxError: invalid syntax

\end{spverbatim}  \\ \hline
\end{longtable}