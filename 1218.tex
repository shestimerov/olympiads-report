На 28 минуте олимпиады команда Гимназии №1543 ``Суслики'' отправила на проверку решение по задаче A.``Слоники'' на языке программирования GNU C++ 4.7 (Решение №1218), которое получило вердикт ``Неправильный ответ'' (Протокол проверки №1218)
\begin{lstlisting}[language=c++, label=r1218, caption=Решение №1218]
#include <cstdio>
#include <iostream>
#include <cstdlib>
#include <algorithm>

using namespace std;

int main () {
	//freopen("input.txt", "rt", stdin);
	//freopen("output.txt", "wt", stdout);
	bool f = 1;
	int n, ap, af, bp, bf, c, d, p, k, x, y;
	scanf("%d%d%d%d%d", &n, &ap, &bp, &af, &bf);
	//
	if (n == -1) {
		swap(bp, bf);
	}
	c = max(ap, af) - min(ap, af) - 1;
	d = min(bp, bf) - 1;
	p = c - d;
	if (p < 0) {
		f = 0;
	} else {
		k = max(bp, bf) + p;
		if ((af - ap) % k < 0) {
			x = k + ((af - ap) % k);
		} else {
			x = (af - ap) % k;
		}
		if ((bf - bp) % k < 0) {
			y = k + ((bf - bp) % k);
		} else {
			y = (bf - bp) % k;
		}
		//cout << x << ' ' << y << k;
		if (x == y) {
			printf ("%d", k);
		} else {
			f = 0;
		}
	}
	if (f == 0) {
		printf("%d", -1);
	}
	//} else {
		/*if ((max(af, ap) == ap) && (max(bf, bp) == bp)) {
			if ((max(af, ap) - min(af, ap) == max(bf, bp) - min(bf, bp)) {
			}
		} else {*/
	return 0;
}

\end{lstlisting}


Результат проверки решения №1218:

 \begin{longtable}{|p{1cm}|p{2.5cm}|p{1.5cm}|p{2.5cm}|p{1.5cm}|p{5cm}|}\hline
№ теста & Результат & ЦПУ, сек. & Общее время, сек. & Память & Комментарий по результату работы \\ \hline 
1 & OK & 1 & 2 & 286720 & \begin{spverbatim}OK

\end{spverbatim}  \\ \hline
2 & OK & 1 & 2 & 286720 & \begin{spverbatim}OK

\end{spverbatim}  \\ \hline
3 & OK & 1 & 2 & 286720 & \begin{spverbatim}OK

\end{spverbatim}  \\ \hline
4 & WA & 1 & 2 & 286720 & \begin{spverbatim}Answers differ: [1]: out: -1, corr: 7

\end{spverbatim}  \\ \hline
\end{longtable}