На 28 минуте олимпиады команда Гимназии №1543 ``Суслики'' отправила на проверку решение по задаче A.``Слоники'' на языке программирования GNU C++ 4.7 (Решение №1216), которое получило вердикт ``Неправильный ответ'' (Протокол проверки №1216)
\begin{lstlisting}[language=c++, label=r1216, caption=Решение №1216]
var
   n,m,c,x,y,x1,y1,h,x2,y2,p,x3,y3,i,j,chmax,chmin,gg:integer;
   ax:array[1..1000]of integer;
   ay:array[1..1000]of integer;
   a1x:array[1..1000]of  integer;
   a1y,ax2,ay2,hx,hy:array[1..1000]of  integer;
   
begin
chmax:=-1;
     x2:=-1;
     y2:=-1;
     x3:=1001;
     y3:=1001;
     chmin:=1001;
     readln(n,m,c);
     for i:=1 to c do begin
         readln(x1,y1);
         ax[i]:=x1;
         ay[i]:=y1;
     end;
     readln(h);
      for i:=1 to h do begin
         readln(x1,y1);
         a1x[i]:=x1;
         a1y[i]:=y1;
     end;
     m:=1;
     gg:=1;
     for i:=1 to h do begin
         for j:=1 to c do begin
             x:=abs(a1x[i] - ax[j]);
               y:=abs(a1y[i] - ay[j]);
             if (y+x)>(y2+x2) then begin
             ax2[j]:=x;
             ay2[j]:=y;
              end;

                  if ax2[m]+ay2[m]>chmax then chmax:=ax2[m]+ay2[m];
                  inc(m);

          end;

             hx[gg]:=chmax;
              inc(gg);
           chmax:=-1;
           m:=1;
      end;
     for m:=1 to h do begin
         if hx[m]<chmin then begin
            p:=m;
            chmin:= hx[m];
         end;
         end;

     writeln(chmin);
     write(p);
end.
\end{lstlisting}

\begin{longtabu} to \linewidth{|p{1cm}|p{2.5cm}|p{1.5cm}|p{2.5cm}|p{1.5cm}|p{5cm}|}\caption{Протокол проверки №3492}\\\hline№ теста & Результат & ЦПУ, сек. & Общее время, сек. & Память & Комментарий по результату работы \\\hline
6 & OK & 6 & 8 & 303104 & \begin{verbatim}
OK


\end{verbatim}\\\hline
7 & OK & 0 & 1 & 303104 & \begin{verbatim}
OK


\end{verbatim}\\\hline
4 & OK & 0 & 1 & 303104 & \begin{verbatim}
OK


\end{verbatim}\\\hline
5 & OK & 0 & 1 & 303104 & \begin{verbatim}
OK


\end{verbatim}\\\hline
2 & OK & 0 & 1 & 303104 & \begin{verbatim}
OK


\end{verbatim}\\\hline
3 & OK & 0 & 1 & 303104 & \begin{verbatim}
OK


\end{verbatim}\\\hline
1 & OK & 0 & 1 & 303104 & \begin{verbatim}
OK


\end{verbatim}\\\hline
10 & WA & 0 & 1 & 303104 & \begin{verbatim}
Answers differ: [1]: out: 1001, corr: 1550


\end{verbatim}\\\hline
8 & OK & 0 & 1 & 303104 & \begin{verbatim}
OK


\end{verbatim}\\\hline
9 & OK & 0 & 1 & 303104 & \begin{verbatim}
OK


\end{verbatim}\\\hline
\end{longtabu}